\documentclass[utf-8,a4paper,11pt]{article}

\usepackage{xeCJK}
\usepackage[top=1in, bottom=1in, left=1.00cm, right=1.00cm]{geometry}
\title{菜鸟也学\LaTeX}
\author{保密}
\date{2021-8-8}

\usepackage{amsmath}
\usepackage{listings}

\usepackage{hyperref}


\usepackage{showexpl}


\begin{document}

\maketitle


\LaTeX 的使用方法

\section{LaTeX的使用方法section}
section下面的文字
\subsection{LaTeX的使用方法subsection}
subsection下面的文字
\subsubsection{LaTeX的使用方法subsubsection}
subsubsection下面的文字
\paragraph{\LaTeX{}的使用方法paragraph}
paragraph 下面的文字
\subparagraph{\LaTeX{} 的使用方法subparagraph}
subparagraph 下面的文字
\subparagraph{\LaTeX{} 的使用方法subparagraph}
subparagraph 下面的文字
\subparagraph{\LaTeX{} 的使用方法subparagraph}
subparagraph 下面的文字

\section{公式}
\subsection{公式的环境}
\begin{itemize}
    \item 行内公式

          行内公式使用\$\$ 进行包围。如$\frac{a}{b}=c/d$ 这是行内公式

    \item 单块行公式
          \begin{enumerate}
              \item 环境1

                    单块行公式使用\verb!\[  \]! 进行包围。如\[\frac{a}{b}=c/d\] 这是单块行公式。
              \item 环境2

                    可以使用\$\$\quad\$\$ 双美元进行包围。如 $$\frac{a}{b}=c/d$$
              \item 环境3

                    也可以使用equation。 但是他会自动带tag
                    \begin{lstlisting}
        \begin{equation}
            \frac{a}{b}=c/d
        \end{quotation}
    \end{lstlisting}

                    如:
                    \begin{equation}
                        \frac{a}{b}=c/d
                    \end{equation}
          \end{enumerate}
    \item 多行块公式
          上面两种都是单行公式。只是又个放在行内,一个独立行显示而已。

          如果要真正的实现多行的公式。就要使用 amsmath 的宏包所提供的环境了。

\end{itemize}

\subsection{多行公式}

多行公式是必备的。以下是常用的例子。参考: \href{https://matnoble.me/tech/latex/multi-line-equations/}{公式}

\subsubsection{写不完的多行}

一行写不下的,要写多行的用multiline
\begin{LTXexample}
    \begin{multline}
        p = 3x^6 + 14x^5y + 590x^4y^2 + 19x^3y^3\\
        + \sin{x} + \cos{y} + \tan{a} + e^{x+y} \\
        - 12x^2y^4 - 12xy^5 + 2y^6 - a^3b^3
    \end{multline}
\end{LTXexample}

\subsubsection{有对齐的的多行}

多行的公式。而不是一行写不下的公式,使用align 环境 以\& 进行对齐
\begin{LTXexample}
\begin{align*}
    X^{ABC}& = \int_a^b x \mathrm{d}x \\
    1\times 2=2=4/2=8/4 & =16/8
\end{align*}

\begin{align}
    a & = b + c\\
      & = d + e \notag
\end{align}
\end{LTXexample}

\begin{align}
    a & = b + c \label{eq:eq1}
    \\[3pt]
      & = d + e  \label{eq:eq2}
    \\[3pt]
      & = d + e \notag
\end{align}

分别交叉引用,依旧是使用label。式 (\ref{eq:eq1}) 和式 (\ref{eq:eq2}) 采用 align 对齐环境。 使用notag取消编号。




% 方法一:
% $$
% \frac{a}{a} =
% \begin{cases}
%     a case\\
%     b case
% \end{cases}
% $$

% $$ f(x)=\left\{
% \begin{aligned}
% x & = & \cos(t) \\
% y & = & \sin(t) \\
% z & = & \frac xy
% \end{aligned}
% \right.
% $$

% 方法二:
% $$ F^{HLLC}=\left\{
% \begin{array}{rcl}
% F_L       &      & {0      <      S_L}\\
% F^*_L     &      & {S_L \leq 0 < S_M}\\
% F^*_R     &      & {S_M \leq 0 < S_R}\\
% F_R       &      & {S_R \leq 0}
% \end{array} \right. $$

% 方法三:
% $$f(x)=
% \begin{cases}
% 0& \text{x=0}\\
% 1& \text{x!=0}
% \end{cases}$$
% ————————————————




\end{document}