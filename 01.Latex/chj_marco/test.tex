\documentclass[utf-8,a4paper,11pt]{article}

\usepackage{xeCJK}
\usepackage[top=1in, bottom=1in, left=2.00cm, right=2.00cm]{geometry}
\title{菜鸟也学\LaTeX}
\author{保密}
\date{2021-8-8}

\usepackage{amsmath}
\usepackage{cases}
\usepackage{empheq}

\usepackage{listings}

\usepackage{hyperref}


\usepackage{showexpl}


\begin{document}

\maketitle


\LaTeX 的使用方法

\section{LaTeX的使用方法section}
section下面的文字,section下面的文字,section下面的文字,section下面的文字,section下面的文字,section下面的文字,section下面的文字,section下面的文字,

\subsection{LaTeX的使用方法subsection}
subsection下面的文字。subsection下面的文字。subsection下面的文字。subsection下面的文字。subsection下面的文字。subsection下面的文字。

\subsubsection{LaTeX的使用方法subsubsection}
subsubsection下面的文字。subsubsection下面的文字。subsubsection下面的文字。subsubsection下面的文字。subsubsection下面的文字。subsubsection下面的文字。

\paragraph{\LaTeX{}的使用方法paragraph}
paragraph 下面的文字。paragraph 下面的文字。paragraph 下面的文字。paragraph 下面的文字。paragraph 下面的文字。

\subparagraph{\LaTeX{} 的使用方法subparagraph}
subparagraph 下面的文字。subparagraph 下面的文字。subparagraph 下面的文字。subparagraph 下面的文字。subparagraph 下面的文字。subparagraph 下面的文字。

\subparagraph{\LaTeX{} 的使用方法subparagraph}
subparagraph 下面的文字

\subparagraph{\LaTeX{} 的使用方法subparagraph}
subparagraph 下面的文字

\section{公式}
\subsection{公式的环境}
\begin{itemize}
    \item 行内公式

          行内公式使用\$\$ 进行包围。如$\frac{a}{b}=c/d$ 这是行内公式

    \item 单块行公式
          \begin{enumerate}
              \item 环境1

                    单块行公式使用\verb!\[  \]! 进行包围。如\[\frac{a}{b}=c/d\] 这是单块行公式。
              \item 环境2

                    可以使用\$\$\quad\$\$ 双美元进行包围。如 $$\frac{a}{b}=c/d$$
              \item 环境3

                    也可以使用equation。 但是他会自动带tag
                    \begin{lstlisting}
        \begin{equation}
            \frac{a}{b}=c/d
        \end{quotation}
    \end{lstlisting}

                    如:
                    \begin{equation}
                        \frac{a}{b}=c/d
                    \end{equation}
          \end{enumerate}
    \item 多行块公式
          上面两种都是单行公式。只是又个放在行内,一个独立行显示而已。

          如果要真正的实现多行的公式。就要使用 amsmath 的宏包所提供的环境了。

\end{itemize}

\subsection{多行公式}

多行公式是必备的。以下是常用的例子。参考: \href{https://matnoble.me/tech/latex/multi-line-equations/}{公式}

主要注意的是: 不想加 tag 可以添加 \verb!\notag! 的标记。当然。任何不加*的环境是不会添加 tag 的。 这个和\LaTeX 的其他环境一样。

\subsubsection{写不完的多行}

一行写不下的,要写多行的用multiline
\begin{LTXexample}
    \begin{multline}
        p = 3x^6 + 14x^5y + 590x^4y^2 + 19x^3y^3\\
        + \sin{x} + \cos{y} + \tan{a} + e^{x+y} \\
        - 12x^2y^4 - 12xy^5 + 2y^6 - a^3b^3
    \end{multline}
\end{LTXexample}

\subsubsection{有对齐的的多行}

多行的公式。而不是一行写不下的公式,使用align 环境 以\& 进行对齐
\begin{LTXexample}
\begin{align*}
    X^{ABC}& = \int_a^b x \mathrm{d}x \\
    1\times 2=2=4/2=8/4 & =16/8
\end{align*}

\begin{align}
    a & = b + c\\
      & = d + e \notag
\end{align}
\end{LTXexample}

\begin{align}
    a & = b + c \label{eq:eq1}
    \\[3pt]
      & = d + e  \label{eq:eq2}
    \\[3pt]
      & = d + e \notag
\end{align}

分别交叉引用,依旧是使用label。式 (\ref{eq:eq1}) 和式 (\ref{eq:eq2}) 采用 align 对齐环境。 使用notag取消编号。

\subsubsection{多列对齐 align}

\begin{LTXexample}
\begin{align*} 
  a &=1 & b &=2 & c &=3 \\ 
  d &=-1 & e &=-2 & f &=-5 
\end{align*}
\end{LTXexample}


\subsubsection{不对齐 gather}

\begin{LTXexample}
  \begin{gather} 
    a = b + c \\ 
    d = e + f + g \\ 
    h + i = j + k \notag \\ 
    l + m = n 
  \end{gather}
\end{LTXexample}

\subsubsection{统一编号 equation}

使用 aligned / gathered 环境,并且依赖 \verb!\begin{equation} \end{equation}!,若不须加编号则使用\verb!\begin{equation*} \end{equation*}!,或者\verb!\[ \]!包裹。

\begin{LTXexample}
\begin{equation}
  \begin{aligned} 
      a &= b + c \\
      d &= e + f + g \\
      h + i &= j + k \\
      l + m &= n
  \end{aligned}
\end{equation}

\begin{equation*}
  \begin{aligned} 
      a &= b + c \\
      d &= e + f + g \\
      h + i &= j + k \\
      l + m &= n
  \end{aligned}
\end{equation*}

\[
  \begin{gathered}
      a = b + c \\
      d = e + f + g \\
      h + i = j + k \\
      l + m = n
  \end{gathered}
\]
\end{LTXexample}

\subsubsection{限定符}

\begin{LTXexample}
  \begin{equation}
    \left\{
      \begin{gathered}
          a_{11} x_{1} + a_{12} x_{2} + a_{13} x_{2} = b_{1}
          \\[3pt]
          a_{21} x_{1} + a_{22} x_{3} + a_{23} x_{3} = b_{2}
      \end{gathered}
    \right.
  \end{equation}
  
  \begin{equation*}
      \left\{
        \begin{aligned}
            a_{11} x_{1} + a_{12} x_{2} + a_{13} x_{2} = b_{1}
            \\[3pt]
            a_{22} x_{3} + a_{23} x_{3} = b_{2}
        \end{aligned}
      \right.
   \end{equation*}
\end{LTXexample}

\subsubsection{array 环境}

\begin{LTXexample}
\[
  |x| = \left\{
    \begin{array}{rl}
      -x & \mbox{if } x < 0,\\ 
      0 & \mbox{if } x = 0,\\ 
      x & \mbox{if } x > 0. 
    \end{array} \right.
\]
\end{LTXexample}

\subsubsection{case 环境}

统一编号:
\begin{LTXexample}
  \begin{equation} |x| =
    \begin{cases}
        -x & \mbox{if } x < 0,\\
        0 & \mbox{if } x = 0,\\
        x & \mbox{if } x > 0.
    \end{cases}
  \end{equation}
\end{LTXexample}


分别编号:
需要在导言区载入 cases 宏包 usepackage{cases},并且放在 amsmath 之后
\begin{LTXexample}
\begin{numcases} {|x| =}
  -x & \mbox{if } x < 0 \label{eq:eq41},\\
  0 & \mbox{if } x = 0,\\
  x & \mbox{if } x > 0.
\end{numcases}

\begin{subnumcases} {\label{eq:eq42} |x| =}
    -x & \mbox{if } x < 0,\\
    0 & \mbox{if } x = 0\label{eq:eq43},\\
    x & \mbox{if } x > 0.
\end{subnumcases}
\end{LTXexample}
式 (\ref{eq:eq41}) 和 式 (\ref{eq:eq42}) 和 式 (\ref{eq:eq43})


\subsubsection{empheq 环境}
其与之前的都不同,它可以从整个公式(组)四周加载定界符。从左端、从右端,甚至是用一个盒子包住整个公式(组)

首先在 导言区 载入 usepackage{empheq}。使用它的框架是这样:
\begin{lstlisting}
\usepackage{empheq}
\begin{empheq}[markup instructions]{AMS env name}
  < content AMS environment >
\end{empheq}
\end{lstlisting}


\begin{LTXexample}
\begin{empheq}[left=\empheqlbrace]{align*}
  E & =mc^2
  \\[3pt]
  Y & = \sum_{n=1}^\infty \frac{1}{n^2}
\end{empheq}
\end{LTXexample}


% 方法一:
% $$
% \frac{a}{a} =
% \begin{cases}
%     a case\\
%     b case
% \end{cases}
% $$

% $$ f(x)=\left\{
% \begin{aligned}
% x & = & \cos(t) \\
% y & = & \sin(t) \\
% z & = & \frac xy
% \end{aligned}
% \right.
% $$

% 方法二:
% $$ F^{HLLC}=\left\{
% \begin{array}{rcl}
% F_L       &      & {0      <      S_L}\\
% F^*_L     &      & {S_L \leq 0 < S_M}\\
% F^*_R     &      & {S_M \leq 0 < S_R}\\
% F_R       &      & {S_R \leq 0}
% \end{array} \right. $$

% 方法三:
% $$f(x)=
% \begin{cases}
% 0& \text{x=0}\\
% 1& \text{x!=0}
% \end{cases}$$
% ————————————————




\end{document}